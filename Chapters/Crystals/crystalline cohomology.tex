\section{Crystalline cohomology}
    \subsection{pd-structures}
        \subsubsection{Divided powers}
            \begin{remark}
                Before we begin, let us note that the notion of pd-structures is well-defined even for noncommutative rings, but we focus on the commutative case in order to be able to establish geometric properties in terms of schemes later on. 
            \end{remark}
            \begin{definition}[Divided power (pd) structures] \label{def: pd_structures}
                Let $R$ be a commutative ring and fix some $R$-ideal $I$. Then, a pd-structure on $R$ is give by a family of maps $\{\gamma_n: I \to I\}_{n \in \N}$ satisfying the following properties, for all $m \in \N$, $x, y \in I$, and $a \in R$
                    \begin{enumerate}
                        \item \textbf{(Raising to the zeroth power):} $\gamma_0(x) = 1$.
                        \item \textbf{(Raising to the first power):} $\gamma_1(x) = x$.
                        \item \textbf{(Addition of exponents):} $\gamma_m(x)\gamma_n(x) = \frac{(m + n)!}{m!n!} \gamma_{m + n}(x)$.
                        \item \textbf{(Multiplication of exponents):} $(\gamma_n \circ \gamma_m)(x) = \frac{(mn)!}{(m!)^n n!} \gamma_{mn}(x)$
                        \item \textbf{(Powers of multiples):} $\gamma_n(ax) = a^n\gamma_n(x)$.
                        \item \textbf{(Binomial expansion):} $\gamma_n(x + y) = \sum_{i=0}^n \gamma_i(x)\gamma_{n - i}(y)$
                    \end{enumerate}
                A ring $R$ equipped with pd-structure $\left(I, \gamma := \{\gamma_n\}_{n \in \N}\right)$ is called a \textbf{pd-ring}. 
            \end{definition}
            \begin{remark}[Well-definiteness of pd-structures]
                Note that all of the expressions are well-defined, since the fractions $\frac{(m + n)!}{m!n!}$ and $\frac{(mn)!}{(m!)^n n!}$ are in fact integers: to show that they are indeed integers, note, respectively, that:
                    $$\frac{(m + n)!}{m!n!} = \binom{m + n}{n} = \binom{m + n}{m}$$
                (which implies that $(m + n)!$ is divisible by $m!n!$) and that:
                    $$\frac{(mn)!}{(m!)^n n!} = \binom{mn}{n}$$
            \end{remark}
            
            \begin{proposition}[Properties of pd-structures]
                Let $(A, I, \gamma)$ be a triple consisting of a $\Z$-torsion-free ring $R$, an ideal $I$ of said ring, and a collection $\gamma := \{\gamma_n\}_{n \in \N}$ of endofunctions $\gamma_n: I \to I$. Then, we have the following for all $x \in I$ and all $n \in \N$:
                    \begin{enumerate}
                        \item If $(I, \gamma)$ is a pd-structure then $n! \gamma_n(x) = x^n$.
                        \item $(I, \gamma)$ is unique as a pd-structure on $R$.
                        \item $(I, \gamma)$ is a pd-structure if and only if $n! \gamma_n(x) = x^n$. 
                        \item $(I, \gamma)$ is a pd-structure on $R$ if and only if there exists a set of generators $x_{\alpha}$ of $I$ such that $x_{\alpha}^n \in n! I$. 
                    \end{enumerate}
            \end{proposition}
                \begin{proof}
                    \noindent
                    \begin{enumerate}
                        \item Clearly:
                            $$0! \gamma_0(x) = 1 = x^0$$
                            $$1! \gamma_1(x) = x = x^1$$
                        and these form our inductive base case. Then, assume that $k! \gamma_k(x) = x^k$ for some $k \in \N$ (which we can do thanks to the established base case and the assumption that $R$ is $\Z$-torsion-free) and note that by the definition of pd-structures, we have the following:
                            $$x^{k + 1} = x^k \cdot x = k!\gamma_k(x) \cdot 1!\gamma_1(x) = k!\frac{(k + 1)!}{k! 1!} \gamma_{k + 1}(x) = (k + 1)!\gamma_{k + 1}(x)$$
                        Thus, we have completed the inductive step, and hence have shown that $n!\gamma_n(x) = x^n$ for all $x \in I$ and all $n \in \N$.
                        \item Suppose that to a fixed $R$-ideal $I$, one can associate two distinct pd-structures $\gamma := \{\gamma_n\}_{n \in \N}$ and $\theta := \{\theta_n\}_{n \in \N}$. Then, according to \textbf{1}, one has the following for all $x \in I$ and all $n \in \N$:
                            $$x^n = n!\gamma_n(x) = n!\theta_n(x)$$
                        which implies that $\gamma_n = \theta_n$ for all $n \in \N$, since $R$ is $\Z$-torsion-free.
                        \item We have already shown in \textbf{1} that $(I, \gamma)$ being a pd-structure implies that $x^n = n!\gamma_n(x)$ for all $n \in \N$ and all $x \in I$. Thus, it suffices to demonstrate that if $x^n = n!\gamma_n(x)$ for all $n \in \N$ and all $x \in I$ then $\gamma := \{\gamma_n\}_{n \in \N}$ is a pd-structure. Let us do this by checking the axioms defining pd-structures one by one.
                            \begin{enumerate}
                                \item \textbf{(Raising to the zeroth power):} $1 = x^0 = 0! \gamma_0(x)$.
                                \item \textbf{(Raising to the first power):} $x = x^1 = 1! \gamma_1(x)$.
                                \item \textbf{(Addition of exponents):} Note that we have the following for all $m, n \in \N$ and all $x \in I$:
                                    $$m!\gamma_m(x) \cdot n!\gamma_n(x) = x^m \cdot x^n = x^{m + n} = (m + n)!\gamma_{m + n}(x)$$
                                and as established above, the fraction $\frac{(m + n)!}{m!n!}$ is actually an integer for all $m, n \in \N$, and thus this implies that:
                                    $$\gamma_m(x)\gamma_n(x) = \frac{(m + n)!}{m!n!}\gamma_{m + n}(x)$$
                                \item \textbf{(Multiplication of exponents):} Consider the following:
                                    $$
                                        \begin{aligned}
                                            \frac{(mn)!}{(m!)^n n!}\gamma_{mn}(x) & = \frac{1}{(m!)^n n!} x^{mn}
                                            \\
                                            & = \frac{1}{(m!)^n n!} \left((-)^n \circ (-)^m\right)(x)
                                            \\
                                            & = \left(\frac{1}{n!}(-)^n \circ \frac{1}{m!}(-)^m\right)(x)
                                            \\
                                            & = (\gamma_n \circ \gamma_n)(x)
                                        \end{aligned}
                                    $$
                                Again, note that everything is well-defined here, since $\frac{(mn)!}{(m!)^n n!}$ is actually an integer.
                                \item \textbf{(Powers of multiples):} This is rather straightforward:
                                    $$(ax)^n = a^n x^n = a^n n! \gamma(ax) = n!\gamma(ax)$$
                                \item \textbf{(Binomial expansion):} Let $x, y$ be elements of $I$ and consider the following for all powers $n \in \N$:
                                    $$
                                        \begin{aligned}
                                            n!\gamma_n(x + y) & = (x + y)^n
                                            \\
                                            & = \sum_{i=0}^n \binom{n}{i} x^i y^{n - i}
                                            \\
                                            & = \sum_{i=0}^n \frac{n!}{(n - i)! i!} x^i y^{n - i}
                                            \\
                                            & = n!\sum_{i=0}^n \gamma_i(x)\gamma_{n - i}(y)
                                        \end{aligned}
                                    $$
                                Thus:
                                    $$\gamma_n(x + y) = \sum_{i=0}^n \gamma_i(x)\gamma_{n - i}(y)$$
                                for all $n \in \N$ and all $x, y \in I$. 
                            \end{enumerate}
                        \item 
                    \end{enumerate}
                \end{proof}
                
            \begin{example}
                \noindent
                \begin{enumerate}
                    \item \textbf{(The trivial pd-structure):} On the zero ideal of any (not even necessary commutative) ring, there exists an obviously trivial pd-structure given by the zero morphism in the abelian category of rings. 
                    \item \textbf{(Divided power structures on $\Z_{(p)}$-algebras):} Let $p$ be a prime and let $R$ be a $\Z_{(p)}$-algebra, i.e. a commutative ring in which all integers not divisible by $p$ are declared to be invertible, and consider the ideal $I := pR$. One can then endow this ideal with a pd-structure $\gamma = \{\gamma_n\}_{n \in \N}$ whose components $\gamma_n$ are given by:
                        $$\forall x \in I: \forall n \in \N: \gamma_n(x) := \frac{1}{n!}x^n$$
                    (which we note, first and foremost, to be well-defined, since every integer not divisible by $p$ is invertible in $R$, and in the event that $n$ is divisible by $p$, the fraction $\frac{x^n}{n!} = \frac{(pa)^n}{n!}$ would still be an integer; one may show this, for instance, by using the fact that within the prime factorisation of $n!$, the exponent of $p$ is $\sum_{r=1}^{+\infty} \left\lfloor \frac{n}{p^r} \right\rfloor$); we leave the verification of the axioms defining pd-structures to the readers (alternatively, one can simply notice that for all $x \in I$ and all $n \in \N$, one has that $x^n = n!\gamma_n(x)$, which we know implies that $\gamma$ is a pd-structure on $I$).
                    \item \textbf{(Divided power structures on $\Q$-algebras):} Given any $\Q$-algebra $R$ and any $R$-ideal $I$, there is an obvious pd-structure $\gamma := \{\gamma_n\}_{n \in \N}$ on $I$ given by:
                        $$\gamma_n(x) := \frac{1}{n!} x^n$$
                \end{enumerate}
            \end{example}
            
            \begin{proposition}
                Let $R$ be a ring and let $(I, \gamma), (J, \delta)$ be a pd-structures on $R$-ideals $I$ and $J$. Then:
                    \begin{enumerate}
                        \item The pd-structures $\gamma$ and $\delta$ agree on $IJ$. In other words, one has a (unique) pd-structure $(IJ, \theta)$ given by:
                            $$(IJ, \theta) = (IJ, \gamma) = (IJ, \delta)$$
                        \item If $\gamma$ and $\delta$ agree on $I \cap J$ then they coincide with the restriction of a (unique) pd-structure $\e$ on $I + J$ down onto $I \cap J$.
                    \end{enumerate}
            \end{proposition}
                \begin{proof}
                    For the sake of reference, let us recall first of all that for any commutative ring $R$ and all $R$-ideals $I$ and $J$, we have the following definitions:
                        $$IJ := \left\{\sum_{1 \leq i, j \leq N} x_iy_j \: \bigg| \: x \in I, y \in y, N \in \N \right\}$$
                        $$I + J := \{x + y \mid x \in I, y \in J\}$$
                        $$I \cap J := \{x \in R \mid (x \in I) \wedge (x \in J)\}$$
                    \begin{enumerate}
                        \item Let $x$ and $y$, respectively, be two arbitrary elements of $I$ and $J$ and consider the following, wherein $n$ is an arbitrary natural number:
                            $$\delta_n(xy) = x^n\delta_n(y) = n!\gamma_n(x) \frac{1}{n!}y^n = \gamma_n(x)y^n = \gamma_n(xy)$$
                        This shows that $\delta$ and $\gamma$ coincide on products of elements of $I$ and $J$. Now, because elements of $IJ$ are finite sums of such products, let us simply consider a sum of two such products:
                            $$\delta_n(xy + x'y') = \sum_{i = 0}^n \delta_i(xy)\delta_{n - i}(x'y') = \sum_{i = 0}^n \gamma_i(xy)\gamma_{n - i}(x'y') = \gamma_n(xy + x'y')$$
                        Thus, $\delta$ and $\gamma$ indeed coincide on finite sums of products of $I$ and of $J$, i.e. on all elements of $IJ$. In other words, they coincide on $IJ$. One can then let $\theta$ be the restriction of $\delta$ (or equivalently, of $\gamma$) down onto $IJ$ (which we note to be a sub-ideal of both $I$ and $J$), and it is necessarily unique by virtue of being a pd-structure.
                        \item Assume firstly that $\gamma$ and $\delta$ conincide with the restriction of a pd-structure $\e$ on $I + J$ down onto $I \cap J$. 
                    \end{enumerate}
                \end{proof}
                
            \begin{proposition}[$p$-powers in pd-rings] \label{prop: p_powers_in_pd_rings}
                Let $p$ be a prime and let $(R, I , \gamma)$ be a pd-ring. Also, assume that $p$ is nilpotent in $R/I$. Then, the ideal $I$ is locally nilpotent if and only if $p$ is nilpotent in $R$. 
            \end{proposition}
                \begin{proof}
                    Suppose first of all, that the ideal $I$ is locally nilpotent, i.e. that for all elements $x \in I$, there exists a natural number $N_x$ such that $x^{N_x} = 0$. Then, we can use the fact that $\gamma$ is a pd-structure on $I$ if and only if $x^n = n!\gamma_n(x)$ for all $x \in I$ and all $n \in \N$; in combination with the local nilpotency hypothesis, this implies that for all $x \in I$, there exists $N_x \in \N$ such that:
                        $$0 = x^{N_x} = N_x!\gamma_{N_x}(x)$$
                \end{proof}
                
        \subsubsection{pd-homomorphisms}
            \begin{definition}[pd-homomorphisms] \label{def: pd_homomorphisms}
                Let $(A, I , \gamma)$ and $(B, J, \delta)$ be two pd-rings and let $\phi: A \to B$ be a ring homomorphism. Then, $\phi$ is a homomorphism of pd-rings if and only if $\phi(I) \subseteq J$ and or all natural numbers $n$, one has the following commutative diagrams:
                    $$
                        \begin{tikzcd}
                        	{I} & {I} \\
                        	{J} & {J}
                        	\arrow["{\phi}"', from=1-1, to=2-1]
                        	\arrow["{\phi}", from=1-2, to=2-2]
                        	\arrow["{\delta}", from=2-1, to=2-2]
                        	\arrow["{\gamma}", from=1-1, to=1-2]
                        \end{tikzcd}
                    $$
                Often, given a pd-homomorphism $\phi: (A, I, \gamma) \to (B, J, \delta)$, one says that $B$ is a pd-algebra over $A$. 
            \end{definition}
            \begin{remark}[Categories pd-algebras]
                Via this definition of pd-homomorphism, for each $\Z_{(p)}$-algebra $A$ (we are consider $\Z_{(p)}$-algebras first because they have been shown to come equipped with canonical pd-sturctures), one gets directly the category of pd-algebras over $A$, denoted by ${}^{A/}\pd\Comm\Alg$. It is a subcategory of ${}^{A/}\Comm\Alg$ that is \textbf{not full}, as not all homomorphisms of (commutative) $A$-algebras are pd-homomorphisms. Note that $\Q$-algebras are $\Z_{(p)}$-algebra (as $\Q \cong \Z_{(p)}[1/p]$), and so categories of pd-algebras over $\Q$-algebras exist in a similar fashion.
                \\
                Note that it is not the case that only pd-algebras over $\Z_{(p)}$ (for some prime $p$) form categories. However, if the base pd-ring is not a $\Z_{(p)}$-algebra, then there need not exist a terminal pd-ring (and as a consequence, no \say{absolute} category of pd-rings like ${}^{\Z_{(p)}/}\Comm\Alg$ or ${}^{\Q/}\Comm\Alg$), as there is no canonical way to associate pd-structures to ideals of general commutative rings.  
            \end{remark}
            
            \begin{proposition}[Completeness of pd-algebra categories]
                Let $p$ be a prime. For any $\Z_{(p)}$-algebra $A$, the category of pd-algebras over $A$ is both complete and cocomplete. Moreover, limits of pd-$A$-algebras agree with those of their underlying commutative $A$-algebras.
            \end{proposition}
                \begin{proof}
                    Because limits can be built out of products and equalisers (see \cite{maclane}, theorem V.2.1), it will suffice to show that ${}^{A/}\pd\Comm\Alg$ has all products and all equalisers.
                    
                    First, let us show that ${}^{A/}\pd\Comm\Alg$ has all products, and to that end, let:
                        $$\left\{\left(B^{(i)}, \b^{(i)}, \gamma^{(i)}\right)\right\}_{i \in I}$$
                    be a small discrete diagram of pd-algebras over $A$ (understood to be equipped with the canonical pd-structure on $\Z_{(p)}$-algebras). Now, consider the ideal $\prod_{i \in I} \b^{(i)}$ of the ring $\prod_{i \in I} B^{(i)}$, and note that because addition and multiplication on products of rings are determined component-wise, there is a natural pd-structure given by:
                        $$\prod_{i \in I} \gamma^{(i)} = \left\{\prod_{i \in I} \gamma_n^{(i)}\right\}_{n \in \N}$$
                    Thus, the category ${}^{A/}\pd\Comm\Alg$ has arbitrary products. 
                    
                    Now, let us prove that ${}^{A/}\pd\Comm\Alg$ has all equalisers, which we can do by checking if any diagram consisting of a pair of parallel pd-homomorphisms of pd-$A$-algebras $\phi, \psi: B \toto C$ has a limit that is also a pd-$A$-algebra.  
                \end{proof}
            \begin{corollary}[Free pd-algebras] \label{coro: free_pd_algebras}
                Let $p$ be a prime and let $A$ be any $\Z_{(p)}$-algebra. The forgetful functor:
                    $$\oblv: {}^{A/}\pd\Comm\Alg \to {}^{A/}\Comm\Alg$$
                does not preserve colimits in general (finite or otherwise), but it does admit a left-adjoint.
            \end{corollary}
                \begin{proof}
                    Let $I$ be the shape of some diagram of pd-algebras over $A$ (again, understood to be equipped with the canonical pd-structure on $\Z_{(p)}$-algebras). As shown above, the colimit taken in ${}^{A/}\pd\Comm\Alg$ over $I$ need not coincide with that taken in ${}^{A/}\Comm\Alg$, and thus the forgetful functor from ${}^{A/}\pd\Comm\Alg$ to ${}^{A/}\Comm\Alg$ need not preserve colimits in general. The rest follows directly. 
                    
                    Now, because limits of pd-algebras are just limits of the underlying commutative algebras, the forgetful functor:
                        $$\oblv: {}^{A/}\pd\Comm\Alg \to {}^{A/}\Comm\Alg$$
                    must preserve limits, and because ${}^{A/}\pd\Comm\Alg$ is locally small (by virtue of being a subcategory of the locally small category ${}^{A/}\Comm\Alg$), complete, and cocomplete (which implies that ${}^{A/}\pd\Comm\Alg$ is presentable), it therefore admits a left-adjoint by the Special Adjoint Functor Theorem \cite[Theorem V.8.2]{maclane}: in essence, we have a well-defined notion of free pd-algebras over a given base commutative ring.
                \end{proof}
                
            \begin{definition}[pd-envelopes] \label{def: pd_envelopes}
                Let $p$ be a prime and fix a $\Z_{(p)}$-algebra. Then, the construction of free pd-$A$-algebras that is left-adjoint to the forgetful functor $\oblv: {}^{A/}\pd\Comm\Alg \to {}^{A/}\Comm\Alg$ (cf. corollary \ref{coro: free_pd_algebras}) shall be called the \textbf{pd-enveloping functor} over $A$; we denote it by:
                    $$\pd: {}^{A/}\Comm\Alg \to {}^{A/}\pd\Comm\Alg$$
                or $\pd_A$ when the base ring $A$ needs emphasis.
            \end{definition}
            
            \begin{definition}[Induced pd-structures]
                Let $(A, I, \gamma)$ be a pd-ring and let $B$ be an $A$-algebra. Then, we say that the pd-structure $\gamma$ extends to a pd-structure $\overline{\gamma}$ on the ideal $IB$ of $B$ if there exists a homomorphism of pd-rings from $(A, I, \gamma)$ to $(B, IB, \overline{\gamma})$. Sometimes, we might refer to pd-structures such as $\overline{\gamma}$ above as induced pd-stuctures. 
            \end{definition}
            
            \begin{proposition}[Existence and uniqueness of induced pd-structures] \label{prop: induced_pd_structures_existence_and_uniqueness}
                Let $(A, I, \gamma)$ be a pd-ring and let $B$ be an $A$-algebra. Then, $\gamma$ extends to a pd-structure if at least one of the following conditions is satisfied:
                    \begin{enumerate}
                        \item $IB = 0$.
                        \item $I$ is a principal ideal.
                        \item $B$ is flat as an $A$-module.
                    \end{enumerate}
                Furthermore, if $\gamma$ does indeed extend to a pd-structure on $IB$, then said induced pd-structure will be unique.
            \end{proposition}
                \begin{proof}
                    \noindent
                    \begin{enumerate}
                        \item \textbf{(Existence):} 
                            \noindent
                            \begin{enumerate}
                                \item We have already seen that the zero ideal of any ring possesses a canonical pd-structure, so if $IB = 0$ then $\gamma$ trivially extends to a (unique) pd-structure on $IB$.
                                \item Now, suppose that $I$ is a principal ideal, say, generated by some element $a \in A$. Suppose also, that $B$ is given by the ring homomorphism $\phi: A \to B$.
                                \item Consider the following short exact sequence of $A$-modules:
                                    $$0 \to I \to A \to A/I \to 0$$
                                Applying the functor $- \tensor_A B$ then gives the following commutative diagram of short sequences of $B$-modules, which we note to both be exact due to the assumption that $B$ is flat over $A$ (and less significantly, due to the fact that the left-adjoint $- \tensor_A B$ preserves finite colimits):
                                    $$
                                        \begin{tikzcd}
                                        	{0} & {I \tensor_A B} & {B} & {A/I \tensor_A B} & {0} \\
                                        	{0} & {IB} & {B} & {B/I} & {0}
                                        	\arrow["{\cong}", from=1-2, to=2-2]
                                        	\arrow[Rightarrow, from=1-3, to=2-3, no head]
                                        	\arrow["{\cong}", from=1-4, to=2-4]
                                        	\arrow[from=1-2, to=1-3]
                                        	\arrow[from=1-3, to=1-4]
                                        	\arrow[from=2-2, to=2-3]
                                        	\arrow[from=2-3, to=2-4]
                                        	\arrow[from=2-4, to=2-5]
                                        	\arrow[from=1-4, to=1-5]
                                        	\arrow[from=1-1, to=1-2]
                                        	\arrow[from=2-1, to=2-2]
                                        \end{tikzcd}
                                    $$
                                
                            \end{enumerate}
                        \item \textbf{(Uniqueness):} Assume that at least one of the conditions guaranteeing that an induced pd-structure $\overline{\gamma}$ exists is satisfied, and suppose to the contrary that there exist two distinct induced pd-structures on $IB$, say $\overline{\gamma}$ and $\tilde{\gamma}$.
                    \end{enumerate}
                \end{proof}
                
        \subsubsection{Quasi-coherent pd-structures}
            \begin{definition}[Quasi-coherent pd-structures] \label{def: quasi_coherent_pd_structures}
                Let $(\calX, \calO)$ be a (small) ringed topos. A \textbf{(quasi-coherent) pd-structure} on a (quasi-coherent) ideal sheaf $\calI \subset \calO$ is thus a family of maps $\{\gamma_n: \calI \to \calI\}_{n \in \N}$ such that $(\calI(U), \gamma_n(U))$ is a pd-structure on $\calO(U)$ in the sense of definition \ref{def: pd_structures}. 
            \end{definition}
                
    \subsection{pd-thickenings}
        \subsubsection{Crystalline sites and topoi}
        
        \subsubsection{Crystals in modules and connections}
            
    \subsection{Drinfeld's stacky approach to crystals}
        \begin{convention}
            Throughout, we fix a perfect field $k$ of characteristic $p > 0$.
        \end{convention}
        
        Let $X$ be a smooth scheme over $\Spec k$, let $X^{\flat}$ denote its tilt, and let $\Witt(X^{\flat})$ be the $p$-adic formal scheme whose underlying topological space is $|\Witt(X^{\flat})| \cong |X^{\flat}|$ and whose structure sheaf is $\Witt(\calO_{X^{\flat}})$. 
    
    \subsection{The de Rham-Witt complex and crystalline cohomology}
    
    \subsection{Derived crystals}