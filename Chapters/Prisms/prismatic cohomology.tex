\section{Prismatic cohomology and prismatisation}
    \subsection{Prisms and prismatic sites}
        \subsubsection{Categories of \texorpdfstring{$\delta$}{}-rings}
            \begin{definition}[$p$-derivations] \label{def: p_derivations}
                Fix a prime $p$. A \textbf{$p$-derivation} on a commutative ring $R$ is a function $\delta: R \to R$ such that:
                    $$\delta(0) = \delta(1) = 0$$
                    $$\delta(xy) = x^p \delta(y) + \delta(x) y^p + p \delta(x) \delta(y)$$
                    $$\delta(x + y) = \delta(x) + \delta(y) - \frac1p \sum_{i = 1}^{p - 1} \binom{p}{i} x^i y^{p - i}$$
            \end{definition}
            \begin{remark}[On the hypothesis of $p$-torsion-freeness]
                Note that because because $p \mid \binom{p}{i} := \frac{p!}{i! (p - i)!}$ for all $1 \leq i \leq p - 1$, the expression $\frac1p \sum_{i = 1}^{p - 1} \binom{p}{i} x^i y^{p - i}$ is well-defined in any commutative ring $R$. If $R$ is furthermore $p$-torsion-free (i.e. the multiplication-by-$p^e$ map $p^e: R \to R: x \mapsto px$ is an injective ring homomorphism for all $e \in \N$), then the fact that:
                    $$p(\delta(x + y) - \delta(x) - \delta(y)) = \sum_{i = 1}^{p - 1} \binom{p}{i} x^i y^{p - i} = (x + y)^p - x^p - y^p$$
                will imply that:
                    $$\delta(x + y) - \delta(x) - \delta(y) = \frac{(x + y)^p - x^p - y^p}{p}$$
                i.e.:
                    $$\delta(x + y) = \delta(x) + \delta(y) + \frac{(x + y)^p - x^p - y^p}{p}$$
                Later on, this alternative form of $\delta(x + y)$ will help us prove that on any $p$-torsion-free ring $R$, $p$-derivations are in bijection with lifts of Frobenii (cf. definition \ref{def: lifts_of_frobenii}).
            \end{remark}
            \begin{definition}[$\delta$-rings] \label{def: delta_rings}
                Fix a prime $p$. A \textbf{$\delta$-ring} is a pair $(R, \delta)$ where $R$ is a commutative ring and $\delta: R \to R$ is a $p$-derivation thereon. Homomorphisms $f: (R, \delta) \to (R', \delta')$ between $\delta$-rings $(R, \delta)$ and $(R', \delta')$, which henceforth shall be known as \textbf{$\delta$-homomorphism}, are simply ring homomorphisms which are $\delta$-equivariant in the sense that $f \circ \delta = \delta' \circ f$; the existence of such $\delta$-homomorphisms yields us a natural category $\Z\-\Comm\Alg^{\delta}$ of $\delta$-rings and $\delta$-homomorphisms between them.
            \end{definition}
            
            \begin{definition}[Lifts of Frobenii] \label{def: lifts_of_frobenii}
                A \textbf{lift of Frobenius} on a commutative ring $R$ is a ring endomorphism $\phi: R \to R$ such that $\phi(x) \equiv x^p \pmod{p}$ for all $x \in R$, i.e. such that the following diagram of commutative rings (wherein the vertical arrows are equal to the canonical reduction-modulo-$p$ quotient map) commutes:
                    $$
                        \begin{tikzcd}
                        	R & R \\
                        	{R/p} & {R/p}
                        	\arrow[two heads, from=1-1, to=2-1]
                        	\arrow[two heads, from=1-2, to=2-2]
                        	\arrow["{\Frob_{R/p}}", from=2-1, to=2-2]
                        	\arrow["\phi", dashed, from=1-1, to=1-2]
                        \end{tikzcd}
                    $$
            \end{definition}
            \begin{proposition}[$\delta$-structures and lifts of Frobenii] \label{prop: p_derivations_and_lifts_of_frobenii}
                Let $R$ be a $p$-torsion-free commutative ring. Then, there is a canonical bijection:
                    $$
                        \left\{\text{$p$-derivations on $R$}\right\}
                        \cong
                        \left\{\text{Lifts of $p^{th}$ power Frobenii on $R$}\right\}
                    $$
            \end{proposition}
                \begin{proof}
                    Observe that for any $\delta$-ring $(R, \delta)$, the map $\phi: R \to R$ given by $\phi(x) := x^p + p\delta(x)$ is a lift of Frobenius: indeed, one has $\phi(x) \equiv x^p \pmod{p}$, which by definition means that the reduction modulo $p$ of the endomorphism $\phi$ is indeed the $p^{th}$ power Frobenius on $R/p$. Conversely, for any commutative ring $R$ with a lift of Frobenius $\phi: R \to R$, consider the map $\delta: R \to R$ determined by $p \delta(x) := \phi(x) - x^p$: if $R$ is $p$-torsion-free, then $p \delta(x) := \phi(x) - x^p$ will imply that $\delta(x) = \frac{\phi(x) - x^p}{p}$; the map $\delta: R \to R$ can then be verified to be a $p$-derivation in accordance with definition \ref{def: p_derivations}. Of course, the assignment of lifts of Frobenii to $p$-derivations and vice versa when $R$ is $p$-torsion-free yields uniquely defined maps, so for such $p$-torsion-free rings $R$, one gets a canonical bijection:
                        $$
                            \left\{\text{$p$-derivations on $R$}\right\}
                            \cong
                            \left\{\text{Lifts of $p^{th}$ power Frobenii on $R$}\right\}
                        $$
                    as claimed.
                \end{proof}
            \begin{remark}
                If $p \in R^{\x}$ (hence $R$ is $p$-torsion-free\footnote{In fact, $p \in R^{\x}$ implies that $R$ is a $\Z[1/p]$-algebra.}, because $p \in R^{\x}$ implies that the multiplication-by-$p$ map is bijective) then actually, every ring endomorphism $\phi: R \to R$ will be a lift of Frobenius. To show that this is true, consider once again $\delta: R \to R$ defined via $\delta(x) := \frac{\phi(x) - x^p}{p}$; since $p \in R^{\x}$, we have as a consequence that $\delta(x) = \phi\left(\frac{x}{p}\right) - \frac{x^p}{p}$. From this, one infers that $\delta$ is a $p$-derivation on $R$, and therefore $\phi$ is a lift of Frobenius due to $R$ lacking $p$-torsion.
            \end{remark}
            \begin{example}[Fermat quotients]
                On $R$ being either $\Z$ or any $\Z_{(p)}$-algebra, one can consider the function $\del_p: R \to R$ given by $\del_p(x) := \frac{x - x^p}{p}$. It is trivial to show that this is a $p$-derivation on $R$, and hence the pair $(R, \del_p)$ is a $\delta$-ring. The attached lift of Frobenius $\phi: R \to R$ is given by $\phi(x) := x^p + p\del_p(x) = x^p + (x - x^p) = x$ (cf. proposition \ref{prop: p_derivations_and_lifts_of_frobenii}), i.e. it is the identity $\id_R$.
            \end{example}
            \begin{definition}[$\delta$-rank] \label{def: delta_rank}
                Let $(R, \delta)$ be a $p$-torsion-free commutative $\delta$-ring. An element $x \in R$ is then said to be of $\delta$-rank $r_0 \geq 1$ (one writes $\delta\-\rank(x) = r_0$) if $\delta^r(x) = 0$ for all $r \geq r_0$; in particular, $\delta\-\rank(x) = 1$ if and only if $\delta(x) = 0$.
            \end{definition}
            \begin{proposition}[Lift of Frobenii act as Frobenii on rank-$1$ elements]
                Let $(R, \delta)$ be a $\delta$-ring and $x \in R$ be an element such that $\delta\-\rank(x) = 1$. Then $\phi(x) = x^p$ with $\phi: R \to R$ being the lift of $p^{th}$ power Frobenius attached to the $p$-derivation $\delta: R \to R$. The converse statement, that $\phi(x) = x^p$ implies that $\delta\-\rank(x) = 1$, is only true when $R$ is $p$-torsion-free.
            \end{proposition}
                \begin{proof}
                    From remark \ref{remark: p_derivations_and_derived_lifts_of_frobenii}, we know that the lift of Frobenius $\phi: R \to R$ attached to the prescribed $p$-derivation $\delta: R \to R$ is determined via $\phi(x) := x^p + p\delta(x)$. But $\delta\-\rank(x) = 1$, so $\delta(x) = 0$, and so it follows that $\phi(x) = x^p$. 
                    
                    For the converse statement, assume first of all that $R$ is $p$-torsion-free. Then because the $p$-derivation $\delta: R \to R$ satisfies $p\delta(x) = \phi(x) - x^p$, which itself implies that $\delta(x) = \frac{\phi(x) - x^p}{p}$ if and only if $R$ is $p$-torsion-free (since only then will the multiplication-by-$p$ map be injective), one has $\delta(x) = \frac{\phi(x) - x^p}{p} = \frac{x^p - x^p}{p} = 0$. By definition, this means that $\delta\-\rank(x) = 1$.
                \end{proof}
            \begin{remark}[$\delta$-structures and derived lifts of Frobenii] \label{remark: p_derivations_and_derived_lifts_of_frobenii}
                A \textbf{derived lift of Frobenius} on a commutative ring $R$ is a ring endomorphism $\phi: R \to R$ whose reduction modulo $p$, which shall be denoted by $\phi_{\F_p}: R/p \to R/p$ is homotopically equivalent to the $p^{th}$-power Frobenius $\Frob_{R/p}: R/p \to R/p$; in other words, for $\phi$ to be a lift of Frobenius on $R$ is for the following diagram to be commutative up to homotopies:
                    $$
                        \begin{tikzcd}
                        	R & R \\
                        	{R/p} & {R/p}
                        	\arrow[two heads, from=1-1, to=2-1]
                        	\arrow[two heads, from=1-2, to=2-2]
                        	\arrow["{\Frob_{R/p}}", from=2-1, to=2-2]
                        	\arrow["\phi", dashed, from=1-1, to=1-2]
                        \end{tikzcd}       
                    $$
                From proposition \ref{prop: p_derivations_and_lifts_of_frobenii}, one then sees that to $p$-derivations $R \to R$ correspond bijectively to paths in the space $\End_{\F_p\-\Comm\Alg^{\simp^{\op}}}\left(R \tensor_{\Z}^{\L} \F_p\right)$ (here $\F_p\-\Comm\Alg^{\simp^{\op}}$ denotes the $(\infty, 1)$-category of simplicial commutative $\F_p$-algebras) between the points $\phi_{\F_p}$ and $\Frob_{R/p}$.
            \end{remark}
            
            Let us now move on to examining the (co)limits of $\delta$-rings as well as universal convstructions involving them. In particular, we shall see that the category $\Z_{(p)}\-\Comm\Alg^{\delta}$ is actually both complete and cocomplete. In fact, that this category is complete and cocomplete is a result of the fact that the evident forgetful functor $\Z_{(p)}\-\oblv^{\delta}: \Z_{(p)}\-\Comm\Alg^{\delta} \to \Z_{(p)}\-\Comm\Alg$ even admits both a left-adjoint and right-adjoint, both of which hold arithmetic significance: colimits and limits can then be \say{reflected} back from $\Z_{(p)}\-\Comm\Alg$, which is both complete and cocomplete.
            \begin{remark}[An initial $\delta$-ring] \label{remark: initial_delta_ring}
                One can rather easily show that the $\delta$-ring $(\Z_{(p)}, \del_p)$ is initial amongst $\Z_{(p)}$-algebras equipped with $p$-derivations (i.e. $\delta$-rings over $\Z_{(p)}$). As such, for any fixed prime $p$, there is a category $\Z_{(p)}\-\Comm\Alg^{\delta}$ of what we shall henceforth refer to as $\delta_p$-algebras, whose objects are $\delta$-rings over $\Z_{(p)}$ and and whose morphisms are $\delta_p$-homomorphisms ($\Z_{(p)}$-algebra homomorphisms which are $\delta$-equivariant in the sense of definition \ref{def: delta_rings}). The \say{absolute} category $\Z\-\Comm\Alg^{\delta}$ of all $\delta$-rings, on the other hand, lacks initial objects. 
            \end{remark}
            \begin{remark}[$\delta$-structures and length-$2$ Witt vectors] \label{remark: p_derivations_and_length_2_witt_vectors} 
                
            \end{remark}
            \begin{proposition}[(Co)limits of $\delta$-rings] \label{prop: (co)limits_of_delta_rings}
                For any fixed prime $p$, the category $\Z_{(p)}\-\Comm\Alg^{\delta}$ of $\delta_p$-algebras admits all colimit and limits, which are computed at the level of the underlying commutative $\Z_{(p)}$-algebras. 
            \end{proposition}
                \begin{proof}
                    Thanks to the Special Adjoint Functor Theorem (cf. \cite{nlab:adjoint_functor_theorem}), the fact that the forgetful functor $\Z_{(p)}\-\oblv^{\delta}: \Z_{(p)}\-\Comm\Alg^{\delta} \to \Z_{(p)}\-\Comm\Alg$ preserves all colimits and all limits that exist in $\Z_{(p)}\-\Comm\Alg^{\delta}$ implies that it admits both a left-adjoint and a right-adjoint. As the category $\Z_{(p)}\-\Comm\Alg$ is both cocomplete and complete, the (co)limits in $\Z_{(p)}\-\Comm\Alg^{\delta}$ first of all exist, and second of all can be computed as (co)limits of the underlying diagrams of commutative $\Z_{(p)}$-algebras; as for the $p$-derivations on these (co)limits, they are as constructed in remark \ref{remark: p_derivations_and_length_2_witt_vectors}.
                \end{proof}
            \begin{corollary}[Quotients and localisations of $\delta$-rings] \label{coro: quotients_and_localisations_of_delta_rings}
                
            \end{corollary}    
            \begin{convention}[Free $\delta$-rings] \label{conv: free_delta_rings}
                Due to the relative nature of $\delta_p$-algebras (cf. remark \ref{remark: initial_delta_ring}), let us hereonafter refer to the left-adjoint of the forgetful functor $\Z_{(p)}\-\oblv^{\delta}$ as the functor of \textbf{free $\delta_p$-algebras} and let us denote it by $\Z_{(p)}\{-\}^{\delta}$.
            \end{convention}
            \begin{proposition}[$\delta$-polynomials] \label{prop: delta_polynomials}
                
            \end{proposition}
                \begin{proof}
                    
                \end{proof}
            \begin{proposition}[Witt vector rings are cofree $\delta$-rings] \label{prop: witt_vector_rings_are_cofree_delta_rings}
                The right-adjoint of the forgetful functor $\Z_{(p)}\-\oblv^{\delta}$ is the $p$-typical Witt vector functor $\Witt(-)$.
            \end{proposition}
                \begin{proof}
                    
                \end{proof}
            
        \subsubsection{Perfect \texorpdfstring{$\delta$}{}-rings}
            \begin{convention}
                Henceforth, we shall start writing $\delta$-rings as triples $(R, \delta, \phi)$ where $\phi$ is the (unique) lift of Frobenius attached to the $p$-derivation $\delta$ (cf. proposition \ref{prop: p_derivations_and_lifts_of_frobenii}). 
            \end{convention}
            \begin{definition}[Perfect $\delta$-rings] \label{def: perfect_delta_rings}
                A $p$-torsion-free $\delta$-ring $(R, \delta, \phi)$ is said to be \textbf{perfect} (or alternatively, $(R, \delta, \phi)$ is \textbf{$\delta$-perfect}) if and only if the lift of Frobenius $\phi: R \to R$ is a ring automorphism. 
            \end{definition}
            \begin{remark}[(Co)perfection of $\delta$-rings] \label{remark: (co)perfection_of_delta_rings}
                It is easy to show that - for a fixed prime $p$, the natural embedding of the category of perfect $\delta_p$-algebras into those of all $\delta_p$-algebras admits both a left-adjoint and a right-adjoint, called \textbf{perfection} and \textbf{coperfection}, given respectively by:
                    $$
                        A^{\perf} :=
                        \colim
                        \left\{
                            \begin{tikzcd}
                                R & R & \cdots
                                \arrow["\phi", from=1-1, to=1-2]
                                \arrow["\phi", from=1-2, to=1-3]
                            \end{tikzcd}
                        \right\}
                    $$
                    $$
                        A^{\coperf} :=
                        \lim
                        \left\{
                            \begin{tikzcd}
                                \cdots & R & R
                                \arrow["\phi", from=1-2, to=1-3]
                                \arrow["\phi", from=1-1, to=1-2]
                            \end{tikzcd}
                        \right\}
                    $$
            \end{remark}
            \begin{lemma}[A criterion for $p$-torsion-freeness] \label{lemma: a_criterion_for_p_torsion_freeness}
                \cite[Lemma 2.28]{bhatt_scholze_prisms} A ring $R$ is $p$-torsion-free if and only if either $R$ is reduced (but not exclusively) or $R$ is perfect (in the sense that $\Frob_R$ is an injective monoid homomorphism).
            \end{lemma}
                \begin{proof}
                    
                \end{proof}
            \begin{definition}[$p$-perfect rings] \label{def: p_perfect_rings}
                A commutative ring $R$ is \textbf{$p$-perfect} (for some prime $p$) if and only if $R/p$ is perfect.
            \end{definition}
            \begin{proposition}
                Fix a prime $p$. There are then the following equivalences of categories:
                    $$
                        \begin{tikzcd}
                        	{\left\{\text{$p$-adically complete $\delta$-perfect $\delta_p$-algebras}\right\}} \\
                        	& {\F_p\-\Comm\Alg^{\perf}} \\
                        	{\left\{\text{$p$-adically complete, $p$-torsion-free, and $p$-perfect $\delta_p$-algebras}\right\}}
                        	\arrow["\Witt"', from=2-2, to=1-1]
                        	\arrow["{R \mapsto R/p}"', from=3-1, to=2-2]
                        	\arrow["\oblv"', from=1-1, to=3-1]
                        \end{tikzcd}
                    $$
            \end{proposition}
                \begin{proof}
                    
                \end{proof}
        
        \subsubsection{Distinguished elements of \texorpdfstring{$\delta$}{}-rings; prisms}
            \begin{definition}[$p$-local rings] \label{def: p_local_rings}
                A commutative ring is said to be \textbf{$p$-local} (for some prime $p$) if and only if $p \in \rad(R)$ (that is, every maximal ideal of $R$ contains $p$ as an element, since $\rad(R) := \bigcap_{\m \in \Spm R} \m$).
            \end{definition}
            \begin{definition}[Distinguished elements of $\delta$-rings] \label{def: distinguished_elements_of_delta_rings}
                An element $a \in R$ of a $\delta$-ring $(R, \delta)$ is said to be \textbf{distinguished} if and only if $\delta(a) \in R^{\x}$.
            \end{definition}
            \begin{proposition}[Detecting distinguishness using lifts of Frobenii] \label{prop: detecting_distinguishness_using_lifts_of_frobenii}
                Fix a prime $p$. An element $a \in R$ of a $p$-local $\delta_p$-algebra $(R, \delta, \phi)$ is distinguished if and only if $\phi(a)$ is also distinguished.
            \end{proposition}
                \begin{proof}
                    Suppose first of all that $\phi(a)$ is distinguished. Per proposition \ref{prop: p_derivations_and_lifts_of_frobenii}, we know that this occurs if and only if $a^p + p\delta(a)$ is distinguished, which in turn means that $\delta(a^p + p\delta(a)) \in R^{\x}$.
                \end{proof}
        
        \subsubsection{Cohomology on prismatic sites}
        
    \subsection{\texorpdfstring{$q$}{}-de Rham cohomology}
    
    \subsection{Prismatic F-crystals and crystalline Galois representations}