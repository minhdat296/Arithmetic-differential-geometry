\section{Prismatic cohomology and prismatisation}
    \subsection{Delta-rings}
        \begin{definition}[$p$-derivations] \label{def: p_derivations}
            \noindent
            \begin{enumerate}
                \item A set-map $\delta_p: B \to B$ is called an \textbf{absolute $p$-derivation} if for all $b, b' \in B$ and all $a \in \Z$, we have:
                    $$\delta_p(a) = 0$$
                    $$\delta_p(b + b') = \delta_p(b) + \delta_p(b') + \left(-\frac1p\sum_{i=0}^{p-1} \binom{p}{i} b^ib'^{p-i}\right)$$
                    $$\delta_p(bb') = b^p\delta_p(b') + \delta_p(b)b'^p + p\delta_p(b)\delta_p(b')$$
                Note that in order for the binomial coefficients $\binom{p}{i} = \frac{p!}{i! (p - i)!}$ (or even $\frac1p \binom{p}{i} = \frac{(p - 1)!}{i! (p - i)!}$ for that matter) to always be well-defined, we will usually want to require $B$ to be a commutative algebra over $\Z_{(p)}$ instead of simply being any commutative ring. 
                \item Let $A$ be a $\Z_{(p)}$-algebra and suppose that $B$ is an $A$-algebra. Then, a \textbf{$p$-derivation on $B$ with coefficients in $A$} is a set-map:
                    $$\delta_p: B \to B$$
                satisfying
                    $$\delta_p(a) = 0$$
                    $$\delta_p(b + b') = \delta_p(b) + \delta_p(b') + \left(-\frac1p\sum_{i=0}^{p-1} \binom{p}{i} b^ib'^{p-i}\right)$$
                    $$\delta_p(bb') = b^p\delta_p(b') + \delta_p(b)b'^p + p\delta_p(b)\delta_p(b')$$
                for all $a \in A$ and $b, b' \in B$.
            \end{enumerate}
        \end{definition}
        \begin{convention}
            For the sake of convenience, let us from now on say that $p$-derivations satisfy $p$-linearity and the $p$-Leibniz rule.
        \end{convention}
        
        \begin{proposition}[$p$-derivations and Frobenius lifts] \label{prop: p_derivations_and_frobenius_lifts}
            This is \cite[Remark 2.2]{bhatt_scholze_prisms}.
            
            If $\delta_p: B \to B$ is a $p$-derivation, then the endomorphism:
                $$\phi^p: B \to B: b \mapsto b^p + p\delta_p(b)$$
            will be a Frobenius lift on $B$. Conversely, if we have any Frobenius lift $\phi^p: B \to B$, and if the ring $B$ is $p$-torsion-free (i.e. $p$ is not a zero-divisor in $B$), then we will get a $p$-derivation $\delta_p$ on $B$:
                $$\delta_p: B \to B: b \mapsto \frac{\phi^p(b) - b^p}{p}$$
            We say that the $p$-derivation $\delta_p$ and the lift of Frobenius $\phi^p$ are attached to one another.
        \end{proposition}
            \begin{proof}
                Firstly, suppose that $\delta_p$ is a $p$-derivation on $B$. Reducing modulo $p$ the map:
                    $$\phi^p: B \to B: b \mapsto b^p + p\delta_p(b)$$
                clearly gives the $p^{th}$-power Frobenius, and so $\phi^p$ is a Frobenius lift. 
                \\
                Conversely, suppose that $\phi^p: B \to B$ is a Frobenius lift, and that $B$ is $p$-torsion-free. Then, it is simply a matter of manually checking the axioms defining a $p$-derivation $\delta_p$ on $B$. Note that we require $B$ to be $p$-torsion-free so that the expression for $\delta_p$ would be well-defined in $B$.
            \end{proof}
        \begin{example}[Examples of $p$-derivations] \label{example: p_derivations}
            \noindent
            \begin{enumerate}
                \item A prototypical example of a $p$-derivation is the Fermat quoient $\del_p$ on $\Z$. This is the set-map given by:
                    $$\del_p z := \frac{z - z^p}{p}$$
                for all $z \in \Z$. We can see that it is attached to the Frobenius lift:
                    $$\phi^p(z) := z^p + p \frac{z - z^p}{p} = z$$
                i.e. $\phi^p = \id_{\Z}$. Note that $\del_p z$ is indeed an integer for any $z \in \Z$, as Fermat's little theorem tells us that:
                    $$z^p \equiv z \pmod{p}$$
                \item One could also extend the definition of the Fermat quotient to any $p$-torsion-free commutative ring $A$ (and especially, any $\Q$-algebra $A$; note that $\Q = \Z_{(p)}[1/p]$), which we will denote by $\del_{p,A}$. For any element $a \in A$, we have:
                    $$\del_{p,A}a = \frac{a - a^p}{p}$$
                and the lift of Frobenius is $\id_A$.
            \end{enumerate}
        \end{example}
            
        \begin{proposition}
            Let $\delta_p$ be a $p$-derivation on some ring $B$, and let $I$ be an ideal of $B$ such that:
                $$\delta_p(I) \subseteq I$$
            Then $\delta_p(I^n) \subseteq I^n$ for all $n \in \N$. 
        \end{proposition}
            \begin{proof}
                This is an immediate consequence of the definition of $p$-derivations.
            \end{proof}
            
        \begin{definition}[$\delta$-rings] \label{def: delta_rings}
            This is \cite[Definition 2.1]{bhatt_scholze_prisms}.
            \begin{enumerate}
                \item A ring equipped with a $p$-derivation is called a \textbf{$\delta$-ring}, or when $p$ is not fixed, a $\delta_p$-ring. A morphism of $\delta$-rings is a homomorphism $B \to B'$ of rings that commute with the $p$-derivations $\delta_p$ and $\delta'_p$ on them, i.e. one gets the following commutative diagram in $\Sets$:
                    $$
                        \begin{tikzcd}
                            B' \arrow[r, "\delta_p'"]         & B'          \\
                            B \arrow[u] \arrow[r, "\delta_p"] & B \arrow[u]
                        \end{tikzcd}
                    $$
                \item Relatively, one could define \textbf{$A$-$\delta$-algebras} as ring homomorphisms $A \to B$, with $A$ a $\Z_{(p)}$-algebra, equipped with $p$-derivations $\delta_p: B \to B$ on $B$ with coefficients in $A$. In other words, $A$-$\delta$-algebras are commutative diagrams in $\Sets$ as follows:
                    $$
                        \begin{tikzcd}
                            B \arrow[r, "\delta_p"]               & B           \\
                            A \arrow[u] \arrow[r, "{\del_{p,A}}"] & A \arrow[u]
                        \end{tikzcd}
                    $$
                A morphism of $A$-$\delta$-algebras are commutative diagrams in $\Sets$ of the following form:
                    $$
                        \begin{tikzcd}
                            B' \arrow[r, "\delta_p'"]         & B'          \\
                            B \arrow[u] \arrow[r, "\delta_p"] & B \arrow[u] \\
                            A \arrow[u] \arrow[r, "\del_{p, A}"]             & A \arrow[u]
                        \end{tikzcd}
                    $$
                \item $\delta$-rings form a category, which we will denote by $\delta\Cring$, or when the prime $p$ is not clear from the context, $\delta_p\Cring$, whose objects are $p$-derivations:
                    $$\delta_p: B \to B$$
                and whose morphisms are commutative diagrams in $\Sets$:
                    $$
                        \begin{tikzcd}
                            B' \arrow[r, "\delta_p'"]         & B'          \\
                            B \arrow[u] \arrow[r, "\delta_p"] & B \arrow[u]
                        \end{tikzcd}
                    $$
            \end{enumerate}
        \end{definition}
        
        \begin{remark}[Arithmetic and algebraic derivations] \label{remark: arithmetic_and_algebraic_derivations}
            \noindent
            \begin{enumerate}
                \item Note that when $\chara B \not = p$, $p$-derivations are not derivations in the usual sense. In characteristic $p$, we could use Fermat's little theorem to see that:
                $$\delta_p(b + b') = \delta_p(b) + \delta_p(b')$$
                $$\delta_p(bb') = b^p\delta_p(b') + \delta_p(b)b'^p \equiv b\delta_p(b') + \delta_p(b)b' \pmod{p}$$
                The first equation also implies that $\delta_p$ is $\Z$-linear in characteristic $p$. Thus, when $\chara B = p$, $p$-derivations are actually derivations, or $\Z$-derivations for that matter, in the usual sense.
                \item More generally, if $B$ is an $A$-algebra (for some commutative $\Q$-algebra $A$), then $p$-derivations on $B$ with coefficients in $A$ are derivations in the usual sense if and only if $\chara B = p$.
                \item More functorially, we recognise that a $\delta_p$-ring $B$ in characteristic $p$ is simply an $\F_p$-algebra, and so $p$-derivations on $\F_p$-algebras are just derivations in the usual sense. Furthermore, this is not just a $\Z$-derivation, but an $\F_p$-derivation. Relatively, a $p$-derivation $\delta_p$ on $B$ with coefficients in some $\Z\left[\frac1p\right] \tensor_{\Z} \F_p$-algebra $A$ is just an $A$-derivation in the usual sense. Note furthermore that $\Z\left[\frac1p\right] \tensor_{\Z} \F_p \cong \F_p[\frac1p]$. This is due to the fact that the free functor:
                    $$\Z[-]: \Sets \to \Cring$$
                as the left-adjoint of the forgetful functor $\Cring \to \Sets$, commutes with colimits, $-\tensor_{\Z} \F_p$ in this instance; that is to say, we have the following natural isomorphisms:
                    $$\Z[-] \tensor_{\Z} \F_p \cong (\F_p \tensor_{\Z} \Z)[-] \cong \F_p[-]$$
            \end{enumerate}
        \end{remark}
        \begin{example}[$p$-derivations that are (not) actually derivations] \label{example: p_derivations_and_derivations}
            \noindent
            \begin{enumerate}
                \item Consider the algebra $B := \F_p[t]$, with $t \not \in p\Z$. There, $\del_{p, \F_p}$ is defined by:
                    $$\del_{p, \F_p}a = 0$$
                    $$\del_{p, \F_p}(af + a'f') = a\del_{p, \F_p}f + a'\del_{p, \F_p}f'$$
                    $$\del_{p, \F_p}(ff') = f\del_{p, \F_p}f' + (\del_{p, \F_p}f) f'$$
                for all $a, a' \in A$ and $f, f' \in \F_p[t]$.
                \item $p$-derivations on neither $\Z_p$ nor $\Q_p$ are actual derivations, as these rings are of characteristic $0$ 
            \end{enumerate}
        \end{example}
         
        \begin{proposition}[Properties of $\delta\Cring$] \label{prop: (co)limits_of_delta_rings}
            Let $p$ be any prime number. The following statements come from example 2.6 and remark 2.7 in \cite{bhatt_scholze_prisms}.
            \begin{enumerate}
                \item $\delta_p\Cring$ has $(\Z, \del_p)$ as the initial object, with $\del_p$ the Fermat quotient. Due to this, one could define the coslice category ${}^{(A, \del_{p,A})/}\delta_p\Comm\Alg$ for any $\Z_{(p)}$-algebra $A$. 
                \item The category $\delta_p\Cring$ is both complete and cocomplete. Furthermore, said (co)limits commute with the forgetful functor:
                    $$\oblv_p: \delta_p\Cring \to \Cring$$
                \item The forgetful functor:
                    $$\oblv_p: \delta_p\Cring \to \Cring$$
                has both left and right-adjoints. The left-adjoint is of course the free construction, and the right-adjoint is the Witt vector functor $\Witt$; the latter point implies that rings of Witt vectors are naturally $\delta$-rings.
            \end{enumerate}
        \end{proposition}

        \begin{definition}[$p$-(pre)derivations and $p$-Leibniz algebras] \label{def: arithmetic_leibniz_algebras}
            Let $k$ be a \textbf{$p$-torsion-free} ring and let $(\V, \tensor, 1)$ be a monoidal $k$-linear category. 
                \begin{enumerate}
                    \item A left/right-$p$-prederivation on an (not necessarily commutative and unital) algebra $\left(\g, \nabla\right)$ internal to $\V$ is a morphism of objects in $\V$:
                        $$\delta: \g \to \g$$
                    that turns the triple $\left(\g, \nabla, \delta\right)$ into an \textbf{$p$-additive} left/right-$p$-Leibniz algebra. That is to say, we require the following diagram to commute in $\V$:
                        $$
                            \begin{tikzcd}
                            	{\g \tensor \g} & {\g} \\
                            	{\g \tensor \g} & {\g}
                            	\arrow["{\delta}", from=1-2, to=2-2]
                            	\arrow["{\nabla}", from=2-1, to=2-2]
                            	\arrow["{\nabla}", from=1-1, to=1-2]
                            	\arrow["{\delta \tensor \Frob_{\g} + \Frob_{\g} \tensor \delta + p \cdot \delta \tensor \delta}"', from=1-1, to=2-1]
                            \end{tikzcd}
                        $$
                    wherein $\Frob_{\g}$ is the usual $p^{th}$-power map.
                    \item If $\g$ also happens to be a unital algebra (with unit map $\eta: 1 \to \g$), then we require that the following diagram commutes:
                        $$
                            \begin{tikzcd}
                            	{1} & {\g} \\
                            	& {\g}
                            	\arrow["{\delta}", from=1-2, to=2-2]
                            	\arrow["{\eta}", from=1-1, to=1-2]
                            	\arrow["{0}"', from=1-1, to=2-2]
                            \end{tikzcd}
                        $$
                    whererin $0$ is understood to be the additive identity in the $k$-module $\V(1, \g)$. In this situation, we call the quadruple $(\g, \nabla, \delta, \eta)$ a \textbf{$p$-linear} $p$-Leibniz algebra, and specifically, the $p$-prederivation $D$ will be referred to simply as a $p$-derivation.
                \end{enumerate}
        \end{definition}
        \begin{example}
            Let us keep notations as in definition \ref{def: arithmetic_leibniz_algebras}.
            \begin{enumerate}
                \item \textbf{($\delta$-rings)} If we take $k$ to be commutative and $\V$ to be the category of $k$-modules, then we can see that $p$-linear $p$-Leibniz algebras internal to $k\mod$ are just $\delta$-rings (cf. definition \ref{def: delta_rings}).  
                \item \textbf{($p$-Lie algebras)} A $p$-Lie algebra is just a (non-unital and non-associative) $p$-additive $p$-Leibniz algebra internal to a braided symmetric monoidal $k$-linear category whose multiplication is a Lie bracket.
            \end{enumerate}
        \end{example}
        
        \begin{claim}[$p$-Leibniz algebras form categories]
            Let $k$ be a $p$-torsion-free ring and let $(\V, \tensor, 1)$ be a monoidal $k$-linear category. Then, $p$-additive $p$-Leibniz algebras internal to $\V$ form a subcategory that admits that of $p$-linear $p$-Leibniz algebras as a subcategory of its own. We shall denote these two categories, repsectively, by $\delta_p\Alg(\V)$ and $\delta_p\Assoc\Alg(\V)$.
        \end{claim}
            \begin{proof}
                Let $(\g, \nabla, \delta)$ and $(\g', \nabla', \delta')$ be two $p$-additive $p$-Leibniz algebras. Then, let us declare that a morphism of $p$-Leibniz algebras internal to $\V$ is an algebra homomorphism $\phi: \g \to \g'$ (i.e. a morphism satisfying $\phi \circ \nabla = \nabla\ \circ (\phi \tensor \phi)$) such that:
                    $$\phi \circ \delta = \delta' \circ \phi$$
                Then, it will suffice to show that the following diagram commutes:
                    $$
                        \begin{tikzcd}
                        	& {\g' \tensor \g'} & {\g'} \\
                        	& {\g' \tensor \g'} & {\g'} \\
                        	{\g \tensor \g} & {\g} \\
                        	{\g \tensor \g} & {\g}
                        	\arrow["{\phi \tensor \phi}", from=3-1, to=1-2]
                        	\arrow["{\phi}", from=3-2, to=1-3]
                        	\arrow["{\phi}", from=4-2, to=2-3]
                        	\arrow["{\delta'}", from=1-3, to=2-3]
                        	\arrow["{\nabla'}", from=1-2, to=1-3]
                        	\arrow["{\delta' \tensor \Frob_{\g'} + \Frob_{\g'} \tensor \delta' + p \cdot \delta' \tensor \delta'}"', from=1-2, to=2-2]
                        	\arrow["{\nabla'}", from=2-2, to=2-3]
                        	\arrow["{\delta}", from=3-2, to=4-2]
                        	\arrow["{\delta \tensor \Frob_{\g} + \Frob_{\g} \tensor \delta + p \cdot \delta \tensor \delta}"', from=3-1, to=4-1]
                        	\arrow["{\nabla}"', from=4-1, to=4-2]
                        	\arrow["{\nabla}"', from=3-1, to=3-2]
                        	\arrow["{\phi \tensor \phi}", from=4-1, to=2-2]
                        \end{tikzcd}
                    $$
                if we are simply trying to show that additive Leibniz algebras form a subcategory of $\V$. For the second assertion, we will, in addition, need to prove that the following diagram, wherein $\eta$ and $\eta'$ are the unit maps, commutes:
                    $$
                        \begin{tikzcd}
                        	&& {1} & {\g'} \\
                        	{1} & {\g} && {\g'} \\
                        	& {\g}
                        	\arrow["{\phi}", from=2-2, to=1-4]
                        	\arrow["{\phi}", from=3-2, to=2-4]
                        	\arrow["{\eta}" description, from=2-1, to=2-2]
                        	\arrow["{\delta}", from=2-2, to=3-2]
                        	\arrow["{\delta'}", from=1-4, to=2-4]
                        	\arrow["{\eta'}" description, from=1-3, to=1-4]
                        	\arrow[Rightarrow, from=2-1, to=1-3, no head]
                        	\arrow["{0}"', from=2-1, to=3-2]
                        	\arrow["{0}"', from=1-3, to=2-4]
                        \end{tikzcd}
                    $$
                To these ends, consider the following:
                    $$
                        \begin{aligned}
                            \phi \circ \delta \circ \nabla & = \phi \circ \nabla \circ \left(\delta \tensor \Frob_{\g} + \Frob_{\g} \tensor \delta + p \cdot \delta \tensor \delta\right)
                            \\
                            & = \nabla' \circ (\phi \tensor \phi) \circ \left(\delta \tensor \Frob_{\g} + \Frob_{\g} \tensor \delta + p \cdot \delta \tensor \delta\right)
                            \\
                            & = \nabla' \circ \left((\phi \circ \delta) \tensor \phi + \phi \tensor (\phi \circ \delta) + p \cdot (\phi \circ \delta) \tensor (\phi \circ \delta)\right)
                            \\
                            & = \nabla' \circ \left((\delta' \circ \phi) \tensor \phi + \phi \tensor (\delta' \circ \phi) + p \cdot (\delta' \circ \phi) \tensor (\delta' \circ \phi)\right)
                            \\
                            & = \nabla' \circ \left(\delta' \tensor \Frob_{\g'} + \Frob_{\g'} \tensor \delta' + p \cdot \delta' \tensor \delta'\right) \circ (\phi \tensor \phi)
                            \\
                            & = \delta' \circ \nabla' \circ (\phi \tensor \phi)
                        \end{aligned}
                    $$
                and in the unital case, also the following:
                    $$
                        \begin{aligned}
                            \delta' \circ \phi \circ \eta & = \delta' \circ \eta'
                            \\
                            & = \delta' \circ 0
                            \\
                            & = 0
                            \\
                            & = \phi \circ 0
                            \\
                            & = \phi \circ \delta \circ \eta
                        \end{aligned}
                    $$
                By matching the terms in these equations with composition of arrows in the preceding two diagrams, we can see that the diagrams indeed commute.
            \end{proof}
    
        \begin{proposition}[Properties of categories of Leibniz algebras]
            Let $k$ be a $p$-torsion-free ring and let $(\V, \tensor, 1)$ be a monoidal $k$-linear category. Also, we shall be writing $\delta_p\Assoc\Alg(\V)$ for the category of associative (and unital) $p$-linear $p$-Leibniz algebras internal to $\V$. 
                \begin{enumerate}
                    \item $\delta_p\Assoc\Alg(\V)$ has $(k, \del_p)$ as the initial object, with $\del_p$ the Fermat quotient. Due to this, one could define the coslice category $\delta_p\Assoc\Alg(\V)_A$ for any $k$-algebra $A$. Then, obviously, $(A, \del_{p,A})$ is initial as an object of $\delta_p\Assoc\Alg(\V)_A$.
                    \item The category $\delta_p\Assoc\Alg(\V)$ is both complete and cocomplete. Furthermore, said (co)limits commute with the forgetful functor:
                        $$U_p: \delta_p\Assoc\Alg(\V) \to \Assoc\Alg(\V)$$
                    \item The forgetful functor:
                        $$U_p: \delta_p\Assoc\Alg(\V) \to \Assoc\Alg(\V)$$
                    has a left-adjoint, namely the free construction.
                \end{enumerate}
        \end{proposition}
        
    \subsection{Prisms and prismatic sites}
        
    \subsection{Cohomology on prismatic sites}