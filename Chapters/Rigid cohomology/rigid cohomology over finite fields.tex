\section{Rigid cohomology for schemes and algebraic stacks over \texorpdfstring{$\F_p$}{}}
    \begin{convention} \label{conv: rigid_analytic_varieties_non_archimedean_ground_field}
        Henceforth, we fix a \textit{complete} non-archimedean field $(K, |-|)$ valuation ring $K^{\circ} := \{x \in K \mid |x| \leq 1\}$ and residue field $k := K^{\circ}/\m_K$, where $\m_K := \{x \in K \mid |x| < 1\}$ is the unique maximal ideal of $K^{\circ}$. In addition, we will be caring about the value group of $K$ (or rather, of $K^{\circ}$), which comes from the valuation $v: K^{\x} \to \R$ that determines the non-archimedean ultra-metric $|-| := \exp(v(-))$, and is given by $|K^{\x}| := \im v$ (note that this is well-defined because $v: K^{\x} \to \R$ is a group homomorphism); this is a totally ordered abelian group whose ordering is inherited from the archimedean ordering of $(\R, \leq)$.
    \end{convention}

    \subsection{Overconvergent sites}
        \subsubsection{Overconvergent schemes}
            \begin{definition}[Overconvergent schemes] \label{def: overconvergent_schemes}
                An \textbf{overconvergent scheme} over $\Spf K^{\circ}$ is a diagram of adic spaces over $\Spa \Z$:
                    $$
                        \begin{tikzcd}
                        	V & {P_{\eta}} \\
                        	X & P
                        	\arrow["{i_{X \to P}}", hook, from=2-1, to=2-2]
                        	\arrow["{\lambda_{V \to P_{\eta}}}", from=1-1, to=1-2]
                        	\arrow[from=1-2, to=2-2]
                        \end{tikzcd}
                    $$
                wherein $i_{X \to P}: X \hookrightarrow P$ is a Zariski-closed immersion of a locally algebraic $k$-scheme $X$ into a formal scheme $P$ that is topologically of finite type over $\Spf K^{\circ}$, and $\lambda_{V \to P_{\eta}}$ is some morphism of Berkovich spaces over $K$. When there is no risk of confusion, we might only refer to a given overconvergent scheme via a triple $(X, P, V)$. 
                
                Morphisms of overconvergent schemes are the obvious commutative diagrams of adic spaces. With these morphisms, one can construct a natural category of overconvergent schemes that we shall denote by $\Rig_{/(k, K^{\circ}, K)}$ or simply $\Rig_{/K^{\circ}}$ for brevity.
            \end{definition}
            \begin{remark}
                Observe first of all, that the category $\Rig_{/(k, K^{\circ}, K)}$ as defined in definition \ref{}
            \end{remark}
        
        \subsubsection{Overconvergent modules}
        
        \subsubsection{Overconvergent stacks and their overconvergent sites}
    
    \subsection{Isocrystals}
    
    \subsection{Comparison with crystalline cohomology}